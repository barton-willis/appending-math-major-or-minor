\documentclass[10pt]{article}
\usepackage{mathptmx}
\usepackage{graphicx}
\usepackage{xcolor}
\usepackage{hyperref}
\usepackage{comment}

\usepackage{enumitem}

\setlist{nosep}

\newenvironment{mypar}[2]
   {\begin{list}{}%
     {\setlength\leftmargin{#1}
     \setlength\rightmargin{#2}}
     \item[]}
   {\end{list}}

% see https://www.unk.edu/ccr/marketing-advertising/branding-and-identity-marks/files/UNK-graphics-standards-quick-guide.pdf

\definecolor{unkblue}{HTML}{002F6C}
\definecolor{unkgold}{HTML}{CC8A00}
\pagestyle{empty}

\usepackage[symbol]{footmisc}

\newcommand{\calcone}{\textbf{MATH 115} Calculus I with Analytic Geometry (fall and spring) \dotfill 5 credits}

\newcommand{\calctwo}{\textbf{MATH 202} Calculus II with Analytic Geometry (fall and spring, prerequisite: MATH 115) \dotfill 5 credits }

\newcommand{\foundations}{\textbf{MATH 250} Foundations of Math (fall and spring, prerequisite: MATH 115)  \dotfill 3 credits}

\newcommand{\calcthree}{\textbf{MATH 260} Calculus III  (fall and spring, prerequisite: MATH 202) \dotfill 5 credits}

\newcommand{\linear}{\textbf{MATH 440} Linear Algebra (spring only, prerequisite: MATH 115) \dotfill 3 credits}

\newcommand{\discrete}{\textbf{MATH 413} Discrete Mathematics  (fall only, prerequisite: MATH 250)\dotfill 3 credits}

\newcommand{\statistics}{\textbf{STAT 441} Probability and Statistics (spring only, prerequisite: MATH 260)  \dotfill  3 credits}

\newcommand{\diffeq}{\textbf{MATH 305}	Differential Equations (spring only, prerequisite: MATH 260) \dotfill 	3 credits}

\newcommand{\abstractalgebra}{\textbf{MATH 350}	Abstract Algebra (spring only, prerequisite: MATH 250) \dotfill 	3 credits}

\newcommand{\complex}{\textbf{MATH 365}	Complex Analysis (spring only,  prerequisite: MATH 260) \dotfill 3 credits}

\newcommand{\advancedcalc}{\textbf{MATH 460}	Advanced Calculus I  (fall only,   prerequisite: MATH 250) \dotfill 3 credits}

\newcommand{\numerical}{\textbf{MATH 420}   Numerical Analysis   (spring only, prerequisite: MATH 260)\dotfill 3 credits}

\newcommand{\collegegeometry}{\textbf{MATH 310}	College Geometry (fall only,  prerequisite: MATH 250) \dotfill 3 credits}

\newcommand{\mathhistory}{\textbf{MATH 400} History of Mathematics (fall only,  prerequisite: MATH 115) \dotfill 3 credits}

\newcommand{\numbertheory}{\textbf{MATH 404} Theory of Numbers (spring only,  prerequisite: MATH 250) \dotfill 3 credits}

\newcommand{\contactbw}{\mbox{Dr.\ Barton Willis}, Department of Mathematics and Statistics,  Discovery Hall (DSCH), Room 368,
\href{mailto:willisb@unk.edu}{willisb@unk.edu} or 308-865-8868.}

\newcommand{\forinfo}[2]{If you would like to discuss the possibility of adding a math {#1} to your {#2}, please contact \contactbw}

\newcommand{\catalog}{2021--2022 }

\newcommand{\myheading}{
\begin{flushleft}
\includegraphics[scale=0.25]{unk-logo}\\
\vspace{0.25in}
 \textcolor{unkblue}{Department of Mathematics and Statistics, Discovery Hall} \\
 %  \emph{\textcolor{unkblue}{Discovery Hall}}\\
  \textcolor{unkblue}{University of Nebraska at Kearney}
\end{flushleft}}

\newcommand{\bt}[1]{\textbf{#1}}
%\usepackage[activate={true,nocompatibility},final,tracking=true,kerning=true,factor=1100,stretch=10,shrink=10]{microtype}
\usepackage[activate={true,nocompatibility},final,tracking=true,kerning=true,spacing=true,factor=1100,stretch=10,shrink=10]{microtype}
\usepackage[american]{babel}
\usepackage[letterpaper, margin=0.65in]{geometry}


\begin{document}

\myheading


\subsection*{\textbf{\textcolor{unkblue}{Computer Science Comprehensive plus Math Minor\footnote[1]{This plan is up to date for the \catalog catalog. Also, the designation of a fall or spring class is anticipated, but it is subject to change.
}}}}


Adding a math minor to your CS degree is one of the most beneficial degree combinations for a Computer Science major, especially if you are interested in machine learning or doing research.  Your math minor will give you a good foundation in set theory, logic, and discrete probability, as well providing you with a high level of mathematical understanding. The curriculum guide for the American Association of Computing Machinery (ACM) deems  this
content  as essential for  a Computer Science education.

 By carefully choosing your elective classes for the Computer Science Comprehensive degree, you can earn a minor in mathematics by taking only one three credit mathematics class beyond the mathematics classes required for your major.  This guide shows you how to choose the electives in your program to also earn a math minor.

\forinfo{minor}{Computer Science Comprehensive degree}

\vspace{-0.1in}

\subsubsection*{\textcolor{black}{For  your General Studies LOPER 4 requirement take}}
\begin{itemize}
\item  \calcone
\end{itemize}

\vspace{-0.1in}
\subsubsection*{\textcolor{black}{For your Computer Science Comprehensive Core Requirements take}}
\begin{itemize}
\item \calctwo
\item \linear
\end{itemize}

\vspace{-0.1in}
\subsubsection*{\textcolor{black}{For your Computer Science electives take}}
\begin{itemize}
\item \foundations
\item \calcthree
\end{itemize}
\vspace{-0.1in}
\begin{mypar}{0.5cm}{0.5cm} Select your remaining credits from CYBR, PHYS,  and MATH. See the catalog for details.
\end{mypar}
\vspace{-0.2in}
\subsubsection*{\textcolor{black}{For your required BS Science-related course requirement take}}
\begin{itemize}
\item \statistics
\end{itemize}
\vspace{-0.1in}
\begin{mypar}{0.5cm}{0.5cm}Alternatively, you may take STAT 241, but STAT 441 will give you  a  better understanding of discrete probability.
\end{mypar}
\vspace{-0.15in}
\subsubsection*{\textcolor{black}{To complete the Mathematics Minor  take}}
\begin{itemize}
\item \discrete
\end{itemize}
\vspace{-0.1in}
\begin{mypar}{0.5cm}{0.5cm}
Alternatively, you make take either MATH 310, 350, 404, or 460, but of these options, MATH 413 is  the class that is most relevant to the CS program.  
\end{mypar}

\vspace{-0.1in}
\subsubsection*{\textcolor{black}{Suggested mathematics course sequence}}

\begin{description}
   \item[\phantom{xxx} First Year] Fall: MATH 115, Spring:  MATH 202
      \item[\phantom{xxx} Second Year]  Fall: MATH 260,  Spring: MATH 250  and MATH 440
     \item[\phantom{xxx} Third Year]  Fall: MATH 413,  Spring: STAT 441
 \end{description}
 \vspace{0.1in}
\noindent Once you declare a mathematics minor, your academic advisor will be happy to help you build a four-year plan for earning a minor in mathematics that fits with your other classes.  If you start on a path toward earning a math minor, but latter decide to earn a  Mathematics Bachelor of Science, all the mathematics and statistics  classes listed here will count toward the Bachelor of Science degree.



\newpage

\myheading



\vspace{-0.1in}
\subsection*{\textbf{\textcolor{unkblue}{Computer Science Comprehensive plus
Mathematics Bachelor of Science\footnote[1]{This plan is up to date for the \catalog catalog. Also, the designation of a fall or spring class is anticipated, but it is subject to change.
}}}}

Adding a math major to your CS degree is one of the most beneficial combinations for a Computer Science degree, especially if you are interested in machine learning or doing research.  Your math major  will give you a solid foundation in set theory, logic, and discrete probability, as well providing you with a high level of mathematical understanding.  The curriculum guide for the American Association of Computing Machinery (ACM) deems  this
content  as essential for  a Computer Science education.

By carefully choosing your optional classes for the Computer Science Comprehensive degree, you can earn a major in mathematics by taking only five three credit classes beyond the mathematics classes required for your major. This guide shows you how to choose the electives in your program to also earn a math major.


\forinfo{major}{Computer Science Comprehensive degree}


\subsubsection*{\textcolor{black}{For  your General Studies LOPER 4 requirement take}}
\begin{itemize}
\item \calcone
\end{itemize}


\subsubsection*{\textcolor{black}{For your mathematics classes required by the CS Major Option take}}
\begin{itemize}
\item \calctwo
\item \linear
\end{itemize}

\subsubsection*{\textcolor{black}{For your Computer Science electives take}}

\begin{itemize}
\item \foundations
\item  \calcthree
\end{itemize}
\begin{mypar}{0.5cm}{0.5cm} Select your remaining credits from CYBR, PHYS,  and MATH. See the catalog for details.
\end{mypar}

\subsubsection*{\textcolor{black}{For your required BS Science-related course requirement take}}
\begin{itemize}
\item \statistics
\end{itemize}
\begin{mypar}{0.5cm}{0.5cm} STAT 441 will count for three credits out of six required MATH or STAT electives.
\end{mypar}

\subsubsection*{\textcolor{black}{For your Math Core requirements take}}

\begin{itemize}
  \item \diffeq
  \item \abstractalgebra
  \item \complex
  \item \advancedcalc
\end{itemize}



\subsubsection*{\textcolor{black}{For your  Math Elective take \emph{one} of}}
\begin{itemize}
\item \discrete
\item \numerical
\end{itemize}
\begin{mypar}{0.5cm}{0.5cm}  There are other course options, but  MATH 413 and MATH 420 are recommend by the department for our
dual CS and Math majors. \end{mypar}

\begin{center} \fbox{
  {\textcolor{unkblue}{Continues on next page.}}}
\end{center}

\subsubsection*{\textcolor{black}{Suggested course sequence}}

\begin{description}
   \item[\phantom{xxx} First  Year] Fall MATH 115, Spring: MATH 202
      \item[\phantom{xxx} Second Year]  Fall: MATH 250 and MATH 260,   Spring: MATH 305 and MATH 440
     \item[\phantom{xxx} Third  Year]  Fall: MATH 413 (unless MATH 420 is planned),  Spring: MATH 350 and STAT 441
      \item[\phantom{xxx} Fourth Year]  Fall: MATH 460,  Spring: MATH 420 (unless have taken MATH 413)
 \end{description}

\begin{mypar}{0.5cm}{0.5cm}  Your mathematics academic advisor will be happy to help you build a four-year plan for earning a Mathematics Bachelor of Science degree.

\end{mypar}
\newpage

%----Cyber Security Operations Comprehensive, Bachelor of Science


\myheading

\subsection*{\textbf{\textcolor{unkblue}{Cyber Security Operations Comprehensive, Bachelor of Science plus Math Minor\footnote[1]{This plan is up to date for the \catalog catalog. Also, the designation of a fall or spring class is anticipated, but it is subject to change.
}}}}

Adding a math minor to your CS degree is one of the most beneficial degree combinations for a Computer Science major, especially if you are interested in machine learning or research.  Your math minor will give you a good foundation in set theory, logic, and discrete probability, as well providing you with a high level of mathematical understanding. The curriculum guide for the American Association of Computing Machinery (ACM) deems  this
content  as essential for  a Computer Science education. This guide shows you how to choose the electives in your program to also earn a math minor.
 
 \forinfo{major}{Cyber Security Operations Comprehensive}


\vspace{-0.1in}

\subsubsection*{\textcolor{black}{For  your General Studies LOPER 4 requirement take}}
\begin{itemize}
\item  \calcone
\end{itemize}

\subsubsection*{\textcolor{black}{To complete the Mathematics Minor take}}

\begin{itemize}
\item \calctwo
\item \foundations
\item \calcthree
\item \discrete
\item \linear
\end{itemize}
\begin{mypar}{0.5cm}{0.5cm}{The classes MATH 413 and MATH 440 are the two that are the most relevant to a CS degree, but for alternatives to these classes, see the catalog.} \end{mypar}

\vspace{-0.1in}
\subsubsection*{\textcolor{black}{Suggested mathematics course sequence}}

\begin{description}
   \item[\phantom{xxx} First Year] Fall: MATH 115, Spring:  MATH 202
      \item[\phantom{xxx} Second Year]  Fall: MATH 260,  Spring: MATH 250  and MATH 440
     \item[\phantom{xxx} Third Year]  Fall: MATH 413
 \end{description}
Once you declare a mathematics minor, your academic advisor will be happy to help you build a four-year plan for earning a minor in mathematics that fits with your other classes.  If you start on a path toward earning a math minor, but latter decide to earn a  Mathematics Bachelor of Science, all the mathematics and statistics  classes listed here will count toward the Bachelor of Science degree.




\newpage

\myheading


\subsection*{\textbf{\textcolor{unkblue}{Physics Comprehensive plus Math Minor\footnote[1]{This plan is up to date for the \catalog catalog. Also, the designation of a fall or spring class is anticipated, but it is subject to change.
}}}}

\noindent Success in Physics classes, especially  Analytic Mechanics (PHYS 402), Electricity \& Magnetism (PHYS 407), and Quantum Mechanics (PHYS 419) require a firm understanding of mathematics. That might help explain why nationwide about one-third of undergraduate physics majors also earn a degree in mathematics. If you choose to attend a graduate program in physics, earning a math minor will help you be successful with the math intense first year core graduate classes.


\forinfo{minor}{Physics Comprehensive}

\subsubsection*{\textcolor{black}{For your General Studies LOPER 4 requirement take}}
\begin{itemize}
\item \calcone
\end{itemize}


\subsubsection*{\textcolor{black}{Physics Comprehensive Math Requirements take}}
\begin{itemize}
 \item \calctwo
 \item \calcthree
 \item \diffeq
\end{itemize}

\subsubsection*{\textcolor{black}{For your Math minor, take}}
\begin{itemize}
\item \foundations
\item \linear
\item \abstractalgebra
\item \advancedcalc
\end{itemize}
There are some options to these classes, but these classes are the most relevant to the physics major. Consult the catalog or your academic
advisor for details.

\subsubsection*{\textcolor{black}{Suggested mathematics course sequence}}

\begin{description}
   \item[\phantom{xxx} First Year] Fall: MATH 115, Spring:  MATH 202, MATH 440
      \item[\phantom{xxx} Second Year]  Fall: MATH 260,  Spring: MATH 305
     \item[\phantom{xxx} Third Year]  Fall: MATH 250,  Spring: MATH 350
     \item[\phantom{xxx} Fourth Year]  Fall: MATH 460
 \end{description}
  \vspace{0.1in}
  

 \noindent Once you declare a mathematics minor, your academic advisor will be happy to help you build a four-year plan. If you start on a path toward earning a math minor, but latter decide to earn a  Mathematics Bachelor of Science, all the math classes listed here will count toward the Bachelor of Science degree.




\newpage

\myheading

\subsection*{\textbf{\textcolor{unkblue}{Adding a Math Minor to your Major\footnote[1]{This plan is up to date for the \catalog catalog. Also, the designation of a fall or spring class is anticipated, but it is subject to change.
}}}}

As you may know, at UNK every Bachelor of Science program of study requires a second
major or minor. So supplementing your major with a math minor is not only a great
way to build your resum\'e, but it also fulfills a degree requirement.

By choosing to take Calculus I with Analytic Geometry (five credits) to satisfy
your LOPER 4 General Studies requirement, you will gain five credits out of a
required twenty-four for a mathematics minor. That leaves only nineteen credits
to complete a math minor.  Of these remaining credits, six are electives, allowing
you to choose from classes ranging from History of Mathematics to Statistics.

%%%JQM

\forinfo{minor}{program of study}


\subsubsection*{\textcolor{black}{For your General Studies LOPER 4 requirement take}}
\begin{itemize}
\item \calcone
\end{itemize}


\subsubsection*{\textcolor{black}{For your Math minor, take}}
\begin{itemize}
  \item \calctwo
  \item \calcthree
\item \foundations
\end{itemize}

\subsubsection*{\textcolor{black}{For your Math minor electives take}}

From the following, take \emph{at least three} credits from
\vspace{0.1in}

\begin{itemize}
\item \collegegeometry
\item \abstractalgebra
\item \numbertheory
\item \discrete
\item \advancedcalc
\end{itemize}

\vspace{0.1in}
\noindent And take \emph{at most  three} credits from
\vspace{0.1in}
\begin{itemize}
\item \diffeq
\item \complex
\item \mathhistory
\item \linear
\item \statistics
\end{itemize}



\subsubsection*{\textcolor{black}{Suggested mathematics course sequence}}

\begin{description}
   \item[\phantom{xxx} First Year] Fall: MATH 115, Spring:  MATH 202
      \item[\phantom{xxx} Second Year]  Fall: MATH 260,  Spring: MATH 250
     \item[\phantom{xxx} Third Year]  Fall: math elective, Spring: math elective
 \end{description}
  \vspace{0.1in}

 \noindent Once you declare a mathematics minor, your academic advisor will be happy to help you build a four-year plan for earning a minor in mathematics.  If you start on a path toward earning a math minor, but latter decide to earn a  Mathematics Bachelor of Science, all the math classes listed here will count toward the Bachelor of Science degree.

   \vspace{0.1in}

\newpage

%----------------
\myheading


\subsection*{\textbf{\textcolor{unkblue}{Adding a Math Major to your Major\footnote[1]{This plan is up to date for the \catalog catalog. Also, the designation of a fall or spring class is anticipated, but it is subject to change.
}}}}

As you may know, at UNK every Bachelor of Science program of study requires a second
major or minor. So supplementing your  program of study with a math major is not only a great
way to build your resum\'e, but it also fulfills a degree requirement.


\forinfo{major}{program of study}

\subsubsection*{\textcolor{black}{For your General Studies LOPER 4 requirement take}}
\begin{itemize}
\item \calcone
\end{itemize}


\subsubsection*{\textcolor{black}{For your Math major requirements take}}
\begin{itemize}
  \item \calctwo
  \item \calcthree
\item \foundations
\item \diffeq
\item \abstractalgebra
\item \complex
\item \linear
\item \advancedcalc
\end{itemize}

\subsubsection*{\textcolor{black}{For your Math major electives take}}

From the following, take \emph{at least six} credits from
\vspace{0.1in}

\begin{itemize}
\item \collegegeometry
\item \mathhistory
\item \numbertheory
\item \discrete
\item \numerical
\item \statistics
\end{itemize}




\subsubsection*{\textcolor{black}{Suggested mathematics course sequence}}

\begin{description}
   \item[\phantom{xxx} First  Year] Fall: MATH 115, Spring:  MATH 202, MATH 250
      \item[\phantom{xxx} Second Year]  Fall: MATH 260,  Spring: MATH 350, math elective
     \item[\phantom{xxx} Third Year]  Fall: MATH 460, Spring: MATH 365, MATH 440
     \item[\phantom{xxx} Fourth Year]  Fall: math elective, Spring: MATH 365, MATH 305
 \end{description}
  \vspace{0.1in}

 \noindent Once you declare a mathematics major, your academic advisor will be happy to help you build a four-year plan for earning a major in mathematics.

   \vspace{0.1in}

\end{document}
