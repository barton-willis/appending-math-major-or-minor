\documentclass[10pt]{article}

\usepackage{mathptmx}
\usepackage{graphicx}
\usepackage{xcolor}

\usepackage{enumitem}

\setlist{nosep}

\newenvironment{mypar}[2]
   {\begin{list}{}%
     {\setlength\leftmargin{#1}
     \setlength\rightmargin{#2}}
     \item[]}
   {\end{list}}

% see https://www.unk.edu/ccr/marketing-advertising/branding-and-identity-marks/files/UNK-graphics-standards-quick-guide.pdf

\definecolor{unkblue}{HTML}{002F6C}
\definecolor{unkgold}{HTML}{CC8A00}
\pagestyle{empty}

\usepackage[symbol]{footmisc}

\newcommand{\calcone}{\textbf{MATH 115} Calculus I with Analytic Geometry (fall and spring) \dotfill 5 credits}

\newcommand{\calctwo}{\textbf{MATH 202} Calculus II with Analytic Geometry (fall and spring, prerequisite: MATH 115) \dotfill 5 credits }

\newcommand{\foundations}{\textbf{MATH 250} Foundations of Math (fall and spring, prerequisite: MATH 115)  \dotfill 3 credits}

\newcommand{\calcthree}{\textbf{MATH 260} Calculus III  (fall and spring, prerequisite: MATH 202) \dotfill 5 credits}

\newcommand{\linear}{\textbf{MATH 440} Linear Algebra (spring only, prerequisite: MATH 115) \dotfill 3 credits}

\newcommand{\discrete}{\textbf{MATH 413} Discrete Mathematics  (fall only, prerequisite: MATH 250)\dotfill 3 credits}

\newcommand{\statistics}{\textbf{STAT 441} Probability and Statistics (spring only, prerequisite: MATH 260)  \dotfill  3 credits}

\newcommand{\diffeq}{\textbf{MATH 305}	Differential Equations (spring only, prerequisite: MATH 260) \dotfill 	3 credits}

\newcommand{\abstractalgebra}{\textbf{MATH 350}	Abstract Algebra (spring only, prerequisite: MATH 250) \dotfill 	3 credits}

\newcommand{\complex}{\textbf{MATH 365}	Complex Analysis (spring only,  prerequisite: MATH 260) \dotfill 3 credits}

\newcommand{\advancedcalc}{\textbf{MATH 460}	Advanced Calculus I  (fall only,   prerequisite: MATH 250) \dotfill 3 credits}

\newcommand{\numerical}{\textbf{MATH 420}   Numerical Analysis   (spring only, prerequisite: MATH 260)\dotfill 3 credits}

\usepackage[activate={true,nocompatibility},final,tracking=true,kerning=true,spacing=true,factor=1100,stretch=10,shrink=10]{microtype}
\usepackage[american]{babel}
\usepackage[letterpaper, margin=0.5in]{geometry}


\begin{document}

\begin{flushleft}
\includegraphics[scale=0.25]{unk-logo}\\
 \emph{\textcolor{unkblue}{Department of Mathematics and Statistics, Discovery Hall}} \\
 %  \emph{\textcolor{unkblue}{Discovery Hall}}\\
  \emph{\textcolor{unkblue}{University of Nebraska at Kearney}}
\end{flushleft}


\subsection*{\textbf{\textcolor{unkblue}{Computer Science Comprehensive plus Math Minor\footnote[1]{This plan is up to date for the 2020-2021 catalog. Also, the designation of a fall or spring class is anticipated, but it is subject to change.
}}}}


Adding a math minor to your CS degree is one of the most beneficial degree combinations for a Computer Science major, especially if you are interested in machine learning or research.  Your math minor will give you a good foundation set theory, logic, and discrete probability, as well providing you with a high level of mathematical understanding.
The curriculum guide for the American Association of Computing Machinery (ACM) deems  these topics and skills  as essential components for  Computer Science education.

 By carefully choosing your optional classes for the Computer Science Comprehensive degree, you can earn a minor in mathematics by taking only one three credit mathematics class beyond the mathematics classes required for your major.  This guide shows you how to choose the electives in your program to also earn a math minor, but it does not list the non-math classes that you need to take.

\subsubsection*{\textcolor{black}{For  your General Studies LOPER 4 requirement take}}
\begin{itemize}
\item  \calcone
\end{itemize}


\subsubsection*{\textcolor{black}{For your mathematics classes required by the CS Major Option take}}
\begin{itemize}
\item \calctwo
\item \linear
\end{itemize}

\subsubsection*{\textcolor{black}{For your required Computer Science electives take}}

\begin{itemize}
\item \foundations
\item \calcthree
\end{itemize}
\vspace{-0.1in}
\begin{mypar}{0.5cm}{0.5cm} This gives eight of the nine required elective credits. For the remaining one credit, you may choose any 300+ level class from CYBR,  MATH, or PHYS.
\end{mypar}
\subsubsection*{\textcolor{black}{For your required BS Science-related course requirement take}}
\begin{itemize}
\item \statistics
\end{itemize}
\vspace{-0.1in}
\begin{mypar}{0.5cm}{0.5cm}Alternatively, you may take STAT 241.But STAT 441 will give you  a  better understanding of discrete probability than STAT 241.
\end{mypar}
\subsubsection*{\textcolor{black}{To complete the Mathematics Minor  take}}

\begin{itemize}
\item \discrete
\end{itemize}
\vspace{-0.1in}
\begin{mypar}{0.5cm}{0.5cm}
Alternatively, you make take any \emph{one} of MATH 310 (College Geometry, fall),  MATH 350 (Abstract Algebra, spring),  MATH 404  (Theory of Numbers, spring), or
MATH 460 (Advanced Calculus I, fall), but of these options, MATH 413 is  the class that is most relevant to the CS program.  \end{mypar}


\noindent


\subsubsection*{\textcolor{black}{Suggested mathematics course sequence}}

\begin{description}
   \item[\phantom{xxx} Freshman year] Fall: MATH 115, Spring:  MATH 202
      \item[\phantom{xxx} Sophomore year]  Fall: MATH 260,  Spring: MATH 250  and MATH 440
     \item[\phantom{xxx} Junior year]  Fall: MATH 413,  Spring: STAT 441
 \end{description}
  \vspace{0.1in}

 \noindent Once you declare a mathematics minor, your academic advisor will be happy to help you build a four-year plan for earning a minor in mathematics that fits with your other classes.  If you start on a path toward earning a math minor, but latter decide to earn a  Mathematics Bachelor of Science, all the mathematics and statistics  classes listed here will count toward the Bachelor of Science degree.

   \vspace{0.1in}

\noindent \textcolor{unkblue}{\emph{If you would like to discuss the possibility of  adding a math minor to your Computer Science Comprehensive degree, please contact \mbox{Dr.\ Willis} (Department of Mathematics and Statistics,  Discovery Hall, Room 386, willisb@unk.edu or 308-865-8868).}}


\newpage

\begin{flushleft}
\includegraphics[scale=0.25]{unk-logo}\\
 \emph{\textcolor{unkblue}{Department of Mathematics and Statistics, Discovery Hall}} \\
  \emph{\textcolor{unkblue}{University of Nebraska at Kearney}}
\end{flushleft}

\vspace{-0.1in}
\subsection*{\textbf{\textcolor{unkblue}{Computer Science Comprehensive plus Mathematics Bachelor of Science\footnote[1]{This plan is up to date for the 2020-2021 catalog. also, the designation of a fall or spring class is anticipated, but it is subject to change.
}}}}

Adding a math major to your CS degree is one of the most beneficial combinations for a Computer Science degree, especially if you are interested in machine learning or doing research.  Your math major  will give you a solid foundation set theory, logic, and discrete probability, as well providing you with a high level of mathematical understanding.   The curriculum guide for the American Association of Computing Machinery (ACM) deems  these topics and skills  as essential components for  Computer Science education.

Our graduates who have  earned dual degrees in CS and mathematics enjoy successful careers in Nebraska as well as through the United States.  To list a few, UNK graduates with CS and math degrees, work as software engineers at well known companies such as The MathWorks, in Natick, Massachusetts, the Walt Disney Company, in New York, New York, and PaymentSpring, in Lincoln, Nebraska.

By carefully choosing your optional classes for the Computer Science Comprehensive degree, you can earn a major in mathematics by taking only five three credit classes beyond the mathematics classes required for your major. This guide shows you how to choose the electives in your program to also earn a math major but it does not list the non-math classes that you need to take.



\subsubsection*{\textcolor{black}{For  your General Studies LOPER 4 requirement take}}
\begin{itemize}
\item \calcone
\end{itemize}


\subsubsection*{\textcolor{black}{For your mathematics classes required by the CS Major Option take}}
\begin{itemize}
\item \calctwo
\item \linear
\end{itemize}

\subsubsection*{\textcolor{black}{For your required Computer Science electives take}}

\begin{itemize}
\item \foundations
\item  \calcthree
\end{itemize}
\begin{mypar}{0.5cm}{0.5cm} This gives eight of the nine required elective credits. For the remaining one credit, you may choose any 300+ level class from CYBR,  MATH, or PHYS.
\end{mypar}

\subsubsection*{\textcolor{black}{For your required BS Science-related course requirement take}}
\begin{itemize}
\item \statistics
\end{itemize}
\begin{mypar}{0.5cm}{0.5cm} STAT 441 will count for three credits out of six required MATH or STAT electives.
\end{mypar}

\subsubsection*{\textcolor{black}{For your Math Core requirements take}}

\begin{itemize}
  \item \diffeq
  \item \abstractalgebra
  \item \complex
  \item \advancedcalc
\end{itemize}

\subsubsection*{\textcolor{black}{For your  Math Elective take \emph{one} of}}
\begin{itemize}
\item \discrete
\item \numerical
\end{itemize}
\begin{mypar}{0.5cm}{0.5cm}  Alternatively, you may take at least three credits from \emph{any} MATH or STAT 300+ level courses. But the classes MATH 413 and MATH 420 are recommend by the department for our
dual CS and Math majors.  The course STAT 441 serves as your remaining Math Elective, for a total of six credits. \end{mypar}


\begin{center} \fbox{
  {\textcolor{unkblue}{For a suggested course sequence, see the next page.}}}
\end{center}
\subsubsection*{\textcolor{black}{Suggested course sequence}}

\begin{description}
   \item[\phantom{xxx} Freshman year] Fall MATH 115, Spring: MATH 202
      \item[\phantom{xxx} Sophomore year]  Fall: MATH 250 and MATH 260,   Spring: MATH 305 and MATH 440
     \item[\phantom{xxx} Junior year]  Fall: MATH 413 (unless MATH 420 is planned),  Spring: MATH 350 and STAT 441
      \item[\phantom{xxx} Senior year]  Fall: MATH 460,  Spring: MATH 420 (unless have taken MATH 413)
 \end{description}

\begin{mypar}{0.5cm}{0.5cm}  Your mathematics academic advisor will be happy to help you build a four-year plan for earning a Mathematics Bachelor of Science  \textcolor{unkblue}{\emph{If you would like to discuss the possibility of  adding a math major to your Computer Science Comprehensive degree, please contact \mbox{Dr.\ Willis} (Department of Mathematics and Statistics,  Discovery Hall, Room 386, willisb@unk.edu or 308-865-8868).}}
\end{mypar}


\newpage

\begin{flushleft}
\includegraphics[scale=0.25]{unk-logo}\\
 \emph{\textcolor{unkblue}{Department of Mathematics and Statistics, Discovery Hall}} \\
 %  \emph{\textcolor{unkblue}{Discovery Hall}}\\
  \emph{\textcolor{unkblue}{University of Nebraska at Kearney}}
\end{flushleft}

\subsection*{\textbf{\textcolor{unkblue}{Physics Comprehensive plus Math Minor\footnote[1]{This plan is up to date for the 2020-2021 catalog. Also, the designation of a fall or spring class is anticipated, but it is subject to change.
}}}}

\noindent Physics classes, especially  Analytic Mechanics (PHYS 402),
Electricity & Magnetism (PHYS 407), and Quantum Mechanics (PHYS 419)
are heavy consumers of mathematics. That might help explain why nationwide, about one-third of undergraduate physics majors also earn a degree in mathematics. If you choose to attend a graduate program in physics, earning a math minor will help you be successful with the math intense first year core graduate classes.

 Adding Mathematics minor to your program of study requires only taking two additional mathematics classes beyond the math classes that are required for your major.


\subsubsection*{\textcolor{black}{For your General Studies LOPER 4 requirement take}}
\begin{itemize}
\item \calcone
\end{itemize}


\subsubsection*{\textcolor{black}{Physics Comprehensive Math Requirements take}}
\begin{itemize}
 \item \calctwo
 \item \calcthree
 \item \diffeq
\end{itemize}

\subsubsection*{\textcolor{black}{For your Math minor, take}}
\begin{itemize}
\item \foundations
\end{itemize}

\subsubsection*{\textcolor{black}{Physics Comprehensive Math, CYBR, or Chemistry Electives, take}}
\begin{itemize}
\item \abstractalgebra
\end{itemize}
\begin{mypar}{0.5cm}{0.5cm} Alternatively, you may choose from MATH 310 (fall only), MATH 404 (spring only), MATH 413 (fall only), or MATH 460 (fall only).  \end{mypar}

\subsubsection*{\textcolor{black}{Suggested mathematics course sequence}}

\begin{description}
   \item[\phantom{xxx} Freshman year] Fall: MATH 115, Spring:  MATH 202
      \item[\phantom{xxx} Sophomore year]  Fall: MATH 260,  Spring: MATH 305
     \item[\phantom{xxx} Junior year]  Fall: MATH 250,  Spring: MATH 350
 \end{description}
  \vspace{0.1in}

 \noindent Once you declare a mathematics minor, your academic advisor will be happy to help you build a four-year plan for earning a minor in mathematics.  If you start on a path toward earning a math minor, but latter decide to earn a  Mathematics Bachelor of Science, all the math classes listed here will count toward the Bachelor of Science degree.

   \vspace{0.1in}

\noindent \textcolor{unkblue}{\emph{If you would like to discuss the possibility of  adding a math minor to your Physics Comprehensive degree, please contact \mbox{Dr.\ Willis} (Department of Mathematics and Statistics,  Discovery Hall, Room 386, willisb@unk.edu or 308-865-8868).}}


\end{document}
