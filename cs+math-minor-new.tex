\documentclass[10pt]{article}
\usepackage{fourier}
\usepackage[T1]{fontenc}
\usepackage{graphicx}
\usepackage{xcolor}
\usepackage{hyperref}
\usepackage{xspace}
\hypersetup{hidelinks}
\usepackage{comment}
\usepackage{enumitem}
\setlist{nosep}

\usepackage{caption}
\usepackage{tabularray}
\UseTblrLibrary{booktabs}
\usepackage{wrapfig,lipsum,booktabs}
% see https://www.unk.edu/ccr/marketing-advertising/branding-and-identity-marks/files/UNK-graphics-standards-quick-guide.pdf

\definecolor{unkblue}{HTML}{002F6C}
\definecolor{unkgold}{HTML}{CC8A00}
\pagestyle{empty}

\usepackage[symbol]{footmisc}

\newcommand{\calcone}{\textbf{MATH 115} Calculus I with Analytic Geometry (fall and spring) \dotfill 5 credits}
\newcommand{\calconeshort}{MATH 115}

\newcommand{\calctwo}{\textbf{MATH 202} Calculus II with Analytic Geometry (fall and spring, prerequisite: MATH 115) \dotfill 5 credits }
\newcommand{\calctwoshort}{MATH 202}

\newcommand{\foundations}{\textbf{MATH 250} Foundations of Math (fall and spring, prerequisite: MATH 115)  \dotfill 3 credits}
\newcommand{\foundationsshort}{MATH 250}

\newcommand{\calcthree}{\textbf{MATH 260} Calculus III  (fall and spring, prerequisite: MATH 202) \dotfill 5 credits}
\newcommand{\calcthreeshort}{MATH 260}

\newcommand{\linear}{\textbf{MATH 280} Linear Algebra (spring only, prerequisite: MATH 115) \dotfill 3 credits}
\newcommand{\linearshort}{MATH 280}

\newcommand{\discrete}{\textbf{MATH 413} Discrete Mathematics  (fall only, prerequisite: MATH 250)\dotfill 3 credits}
\newcommand{\discreteshort}{MATH 413}

\newcommand{\statistics}{\textbf{STAT 441} Probability and Statistics (spring only, prerequisite: MATH 260)  \dotfill  3 credits}
\newcommand{\statisticsshort}{STAT 441}

\newcommand{\statisticsII}{\textbf{STAT 442} Mathematical Statistics   (on demand, prerequisite: STAT 441)  \dotfill  3 credits}
\newcommand{\statisticsIIshort}{STAT 442}

\newcommand{\diffeq}{\textbf{MATH 305}	Differential Equations (spring only, prerequisite: MATH 260) \dotfill 	3 credits}
\newcommand{\diffeqshort}{MATH 305}

\newcommand{\abstractalgebra}{\textbf{MATH 350}	Abstract Algebra (spring only, prerequisite: MATH 250) \dotfill 	3 credits}
\newcommand{\abstractalgebrashort}{MATH 350}

\newcommand{\complex}{\textbf{MATH 365}	Complex Analysis (spring only,  prerequisite: MATH 260) \dotfill 3 credits}
\newcommand{\complexshort}{MATH 365}

\newcommand{\advancedcalc}{\textbf{MATH 460}	Advanced Calculus I  (fall only,   prerequisite: MATH 250) \dotfill 3 credits}
\newcommand{\advancedcalcshort}{MATH 460}

\newcommand{\numerical}{\textbf{MATH 420}   Numerical Analysis   (spring only, prerequisite: MATH 202)\dotfill 3 credits}
\newcommand{\numericalshort}{MATH 420} 

\newcommand{\collegegeometry}{\textbf{MATH 310}	College Geometry (fall only,  prerequisite: MATH 250) \dotfill 3 credits}
\newcommand{\collegegeometryshort}{MATH 310}

\newcommand{\mathhistory}{\textbf{MATH 400} History of Mathematics (fall only,  prerequisite: MATH 115) \dotfill 3 credits}
\newcommand{\mathhistoryshort}{MATH 400}

\newcommand{\numbertheory}{\textbf{MATH 404} Theory of Numbers (spring only,  prerequisite: MATH 250) \dotfill 3 credits}
\newcommand{\numbertheoryshort}{MATH 404} 

\newcommand{\appliedstat}{\textbf{STAT 345} Applied Statistics I (fall only, prerequisite: MATH 123 or MATH 115) \dotfill 3 credits}
\newcommand{\appliedstatshort}{STAT 345}

\newcommand{\physics}{\textbf{PHYS 275 and PHYS 275L}  (corerequisite: \calconeshort) \dotfill 5 credits}

\newcommand{\contactbw}{\mbox{Barton Willis, PhD}, Department of Mathematics and Statistics,  Discovery Hall, Room 368;
email or phone: \href{mailto:willisb@unk.edu}{willisb@unk.edu} or 308-865-8868.}

\newcommand{\forinfo}[2]{If you would like to discuss the possibility of earning a math {#1}, please contact \contactbw}

\newcommand{\catalog}{2023--2024 }

\newcommand{\LOPER}{LOPER\xspace}


\newcommand{\uptodate}{This plan is up-to-date for  the \catalog catalog. The designation of a fall or spring class is 
anticipated, but  is subject to change.}
\newcommand{\myheading}{
\begin{flushleft}
\includegraphics[scale=0.3]{unk-logo}\\
\setcounter{footnote}{0}
\vspace{0.25in}
 \textcolor{unkblue}{Department of Mathematics and Statistics, Discovery Hall} \\
 %  \emph{\textcolor{unkblue}{Discovery Hall}}\\
  \textcolor{unkblue}{University of Nebraska at Kearney}
\end{flushleft}}

\newcommand{\bt}[1]{\textbf{#1}}


\newcommand{\mathminor}{      
\begin{table}[h]
  \caption*{Suggested Math Minor program of study}
   \center
  \begin{tblr}{
      colspec={lll},
      row{1}={font=\bfseries},
      column{1}={font=\itshape},
      row{even}={bg=gray!10},
    }
     Year        & Fall  & Spring   \\
    \toprule
    First & \calconeshort  & \calctwoshort, \linearshort  \\
    Second & STAT 235 & MATH 250  \\
    Third & MATH 414 & \\
    \bottomrule
  \end{tblr}
\end{table}}

\newcommand{\mathminorALT}{
\begin{table}[h]
 \caption*{Suggested Math Minor program of study}
   \center
  \begin{tblr}{
      colspec={lll},
      row{1}={font=\bfseries},
      column{1}={font=\itshape},
      row{even}={bg=gray!10},
    }
     Year        & Fall  & Spring   \\
    \toprule      
      First & \calconeshort & \calctwoshort{}  \\  
      Second &  \calcthreeshort{},  \foundationsshort{}  & \diffeqshort \\
        \bottomrule
  \end{tblr}
   \end{table}}
      
\newcommand{\mathminorALTALT}{
  \begin{table}[h]
   \caption*{Suggested Math Minor program of study}
   \center
  \begin{tblr}{
      colspec={lll},
      row{1}={font=\bfseries},
      column{1}={font=\itshape},
      row{even}={bg=gray!10},
    }
     Year        & Fall  & Spring   \\
    \toprule      
        First & \calconeshort & \calctwoshort{}  \\  
      Second &  \foundationsshort{}  &  math elective \\ 
      Third &  math elective   &  math elective  (if needed) \\ 
            \bottomrule
  \end{tblr}
         \end{table}}
         

      
\newcommand{\mathBSOLD}{
     \begin{center}
         \begin{tabular}[h]{| l | l | l|} 
            \hline
            \textbf{Year}           & \textbf{Fall}         &  \textbf{Spring}  \\ \hline 
            First & \calconeshort{}  & \calctwoshort{} , \linearshort  \\  \hline
            Second &  \calcthreeshort{}, \foundationsshort & \diffeqshort, math elective \\ \hline
            Third & \advancedcalcshort              &  \complexshort{}\\ \hline
            Fourth & math elective &  \abstractalgebrashort  \\ \hline
         \end{tabular}
\end{center}}

\newcommand{\mathBS}{      
\begin{table}[ht]
 \caption*{Suggested Math B.S. program of study}
\center
  \begin{tblr}{
      colspec={lll},
      row{1}={font=\bfseries},
      column{1}={font=\itshape},
      row{even}={bg=gray!10},
    }
     Year        & Fall  & Spring   \\
    \toprule
    First & \calconeshort  & \calctwoshort, \linearshort  \\
    Second & \calcthreeshort{}, \foundationsshort & \diffeqshort, math elective \\
     Third & \advancedcalcshort              &  \complexshort{}\\ \hline
      Fourth & math elective &  \abstractalgebrashort  \\ 
    \bottomrule
  \end{tblr}
\end{table}}


\newcommand{\statMinor}{
 \begin{table}[ht]
  \caption*{Suggested Statistics Minor program of study}
\center
  \begin{tblr}{
      colspec={lll},
      row{1}={font=\bfseries},
      column{1}={font=\itshape},
      row{even}={bg=gray!10},
    }
      Year        & Fall  & Spring   \\
    \toprule    
            First & \calconeshort{}  & \calctwoshort \\ 
            Second &  \appliedstatshort &  elective \\ 
            Third &           &  \statisticsshort \\
       \bottomrule
  \end{tblr}
\end{table}}
\usepackage[final]{microtype}
\usepackage[american]{babel}
\usepackage[letterpaper, margin=0.75in]{geometry}


\begin{document}

\myheading


\subsection*{\textbf{\textcolor{unkblue}{Computer Science Comprehensive plus Math Minor}}}

Adding a math minor to your CS degree is one of the most beneficial 
combinations for a Computer Science major, especially if you are interested 
in machine learning, cryptography, or research. Your math minor will 
give you a good foundation in set theory, logic, and discrete probability, as 
well as providing you with a high level of mathematical understanding.


\forinfo{minor}{Computer Science Comprehensive degree}

\uptodate

\vspace{-0.1in}

\subsubsection*{\textcolor{black}{For your General Studies \LOPER 4 requirement take}}
\begin{itemize}
\item  \calcone
\end{itemize}

\subsubsection*{\textcolor{black}{To complete the Mathematics Minor take}}
\begin{itemize}
\item \calctwo
\item \foundations
\item \linear
\item \discrete
\item \appliedstat
\end{itemize}
\vspace{0.1in}
These are the classes that we recommend for a computer science major.
Your academic advisor can assist you with choosing alternatives 
to \linearshort, \discreteshort, and \appliedstatshort. Additionally,
your academic advisor will help you build a 
four-year plan for earning a minor in mathematics that fits with 
your goals and other classes.  If you start on a path toward earning a math minor, but later decide to 
earn a Mathematics B.S., all the mathematics and statistics  classes listed here 
will count toward the B.S. degree. 
\mathminor


\newpage

\myheading



\vspace{-0.1in}
\subsection*{\textbf{\textcolor{unkblue}{Computer Science Comprehensive plus Mathematics Bachelor of Science}}}

Adding a math major to your CS degree is one of the most beneficial combinations for 
a Computer Science degree, especially if you are interested in machine learning, 
cryptography,   or research.  Your math major  will give you a solid foundation 
in set theory, logic, and discrete probability, 
as well providing you with a high level of mathematical understanding. 



\forinfo{major}{Computer Science Comprehensive degree}

\uptodate

\subsubsection*{\textcolor{black}{For your General Studies \LOPER 4 requirement take}}
\begin{itemize}
\item \calcone
\end{itemize}

\subsubsection*{\textcolor{black}{For your \LOPER 8 requirement take}}

\begin{itemize}
   \item \physics 
\end{itemize}

\subsubsection*{\textcolor{black}{For your Program Specific Requirement take}}

\begin{itemize}
   \item \textbf{ENG 102}, Special Topics in Academic Writing and Research (prerequisite: ENG 101)  \dotfill 3 credits
\end{itemize}

	

\subsubsection*{\textcolor{black}{For your Math Core requirements take}}

\begin{itemize}
   \item \calctwo
   \item \foundations
      \item \calcthree
   \item \linear
   \item \diffeq
  \item \abstractalgebra
  \item \complex
  \item \advancedcalc
\end{itemize}

\subsubsection*{\textcolor{black}{For your  Math Electives take}}
\begin{itemize}
\item \discrete
\item \statistics
\end{itemize}
These are the classes that we recommend for a computer science major.
Your academic advisor can assist you with choosing alternatives 
to \discreteshort{} and \statisticsshort{}. Additionally, your 
academic advisor help you build a four-year plan for earning a 
Mathematics B.S that fits with your goals and other classes.
 \mathBS
%\subsubsection*{\textcolor{black}{Suggested mathematics course sequence}}\vspace{-0.2in}




\newpage

%----Cyber Security Operations Comprehensive, Bachelor of Science


\myheading

\subsection*{\textbf{\textcolor{unkblue}{Cyber Security Operations Comprehensive, Bachelor of Science plus Math Minor}}}

Adding a math minor to your CS degree is one of the most beneficial degree combinations 
for a Computer Science major, especially if you are interested in machine learning, 
cryptography,  or research.  Your math minor will give you a good foundation in 
set theory, logic, and discrete probability, as well providing you with a high level of 
mathematical understanding. 

 
 \forinfo{minor}{Cyber Security Operations Comprehensive}

\uptodate
\vspace{-0.1in}

\subsubsection*{\textcolor{black}{For  your General Studies \LOPER 4 requirement take}}
\begin{itemize}
\item  \calcone
\end{itemize}

\subsubsection*{\textcolor{black}{To complete the Mathematics Minor take}}

\begin{itemize}
\item \calctwo
\item \foundations
\item \linear
\item \appliedstat
\item \discrete
\end{itemize}
\vspace{0.1in}
These are the classes that we recommend for a computer science major.
Your academic advisor can assist you with choosing alternatives 
to \appliedstatshort{} and \discreteshort{}. Additionally, your academic advisor help you build a four-year plan for earning 
a minor in mathematics that fits with your goals and other classes.  
If you start on a path toward earning a math minor, but latter decide to earn a  
Mathematics B.S., all the mathematics and statistics  
classes listed here will count toward the B.S. degree.
%\subsubsection*{\textcolor{black}{Suggested mathematics course sequence}}\vspace{-0.2in}
\mathminor




\newpage

\myheading


\subsection*{\textbf{\textcolor{unkblue}{Physics Comprehensive plus Math Minor}}}

\noindent Success in Physics classes, especially  Analytic Mechanics (PHYS 402), Electricity \& Magnetism (PHYS 407), and Quantum Mechanics (PHYS 419) require a 
firm understanding of mathematics. That might help explain why 
nationwide about one-third of undergraduate physics majors also 
earn a degree in mathematics. If you choose to attend a graduate 
program in physics, earning a math minor will help you be successful 
with the math intense first year core physics graduate classes.


\forinfo{minor}{Physics Comprehensive}

\uptodate

\subsubsection*{\textcolor{black}{For your General Studies \LOPER 4 requirement take}}
\begin{itemize}
\item \calcone
\end{itemize}

\subsubsection*{\textcolor{black}{For your BS Science-related course requirements take}}
\begin{itemize}
\item \calctwo
\item \diffeq
\end{itemize}


\subsubsection*{\textcolor{black}{Physics Comprehensive Math Requirements take}}
\begin{itemize}
 \item \calcthree
\end{itemize}

\subsubsection*{\textcolor{black}{For your Math minor, take}}
\begin{itemize}
\item \foundations
\end{itemize}
\vspace{0.1in}
\noindent Your academic advisor help you build a four-year plan that
fits with your goals and other classes. If you start on a path 
toward earning  a math minor, but latter decide to earn a 
Mathematics B.S., all the math classes listed here will count 
toward the B.S. degree.
\mathminorALT




\newpage

\myheading

\subsection*{\textbf{\textcolor{unkblue}{Adding a Math Minor to your Major}}}

As you may know, at UNK every Bachelor of Science program of study requires a second
major or minor. So supplementing your major with a math minor is not only a great
way to build your r\'esum\'e, but it also fulfills a degree requirement.

By choosing to take Calculus I with Analytic Geometry (five credits) to satisfy
your \LOPER 4 General Studies requirement, you will gain five credits out of a
required twenty-one for a mathematics minor. That leaves only sixteen credits
to complete a math minor.  Of these remaining credits, eight are electives, allowing
you to choose from classes ranging from History of Mathematics to Applied Statistics.


\forinfo{minor}{program of study}

\uptodate

\subsubsection*{\textcolor{black}{For your General Studies \LOPER 4 requirement take}}
\begin{itemize}
\item \calcone
\end{itemize}


\subsubsection*{\textcolor{black}{For your Math minor, take}}
\begin{itemize}
  \item \calctwo

\item \foundations
\end{itemize}

\subsubsection*{\textcolor{black}{For your Math minor electives take}}

From the following, take \emph{at least eight} credits from
\vspace{0.1in}

\begin{itemize}
  \item \calcthree
\item \linear
\item \diffeq
\item \collegegeometry
\item \appliedstat
\item \abstractalgebra
\item \complex
\item \mathhistory
\item \numbertheory
\item \discrete
\item \numerical
\item \statistics
\item \advancedcalc
\end{itemize}

\subsubsection*{\textcolor{black}{Suggested mathematics course sequence}}

\mathminorALTALT

 %\noindent Once you declare a mathematics minor, your academic advisor will be happy to help you build a four-year plan for earning a minor in %mathematics.  If you start on a path toward earning a math minor, but latter decide to earn a  Mathematics Bachelor of Science, all the math classes %listed here will count toward the Bachelor of Science degree.

 %  \vspace{0.1in}

\newpage

%----------------
\myheading


\subsection*{\textbf{\textcolor{unkblue}{Adding a Math Major to your Major}}}

As you may know, at UNK every Bachelor of Science program of study requires a second
major or minor. So supplementing your  program of study with a math major is not only a great
way to build your r\'esum\'e, but it also fulfills a degree requirement.


\forinfo{major}{program of study}

\uptodate

\subsubsection*{\textcolor{black}{For your General Studies \LOPER 4  and  \LOPER 8 requirements, take}}
\begin{itemize}
\item \calcone
   \item \physics 
\end{itemize}
\vspace{-0.1in}
\subsubsection*{\textcolor{black}{For your Program Specific Requirement take}}
\begin{itemize}
   \item \textbf{ENG 102}, Special Topics in Academic Writing and Research (prerequisite: ENG 101)  \dotfill 3 credits
\end{itemize}
\vspace{-0.1in}
\subsubsection*{\textcolor{black}{For your Math major requirements take}}
\begin{itemize}
  \item \calctwo
  \item \calcthree
  \item \linear
\item \foundations
\item \diffeq
\item \abstractalgebra
\item \complex
\item \advancedcalc
\end{itemize}
\vspace{-0.1in}
\subsubsection*{\textcolor{black}{For your Math major electives take}}
From the following, take \emph{at least six} credits from
\begin{itemize}
\item \collegegeometry
\item \mathhistory
\item \numbertheory
\item \discrete
\item \numerical
\item \statistics
\end{itemize}
\vspace{0.0in}
\noindent Your academic advisor help you build a four-year plan for earning a 
major in mathematics that fits with your goals and other classes.

\subsubsection*{\textcolor{black}{Suggested mathematics course sequence}}

\mathBS

\newpage 
\myheading
\subsection*{\textbf{\textcolor{unkblue}{Adding a Statistics Minor to your major}}}



As you may know, at UNK every Bachelor of Science program of study requires a second
major or minor. So supplementing your  program of study with a Statistics major is not only a great
way to build your r\'esum\'e, but it also fulfills a degree requirement.

\forinfo{Statistics Minor}{}

\uptodate

\vspace{-0.1in}

\subsubsection*{\textcolor{black}{For your Calculus Background (including your General Studies \LOPER 4 requirement take}}
\begin{itemize}
\item  \calcone
\item \calctwo
\end{itemize}

\subsubsection*{\textcolor{black}{To complete the Mathematics Statistics take}}

at least three credits from BIOL 305 (BioStatistics), MGT 233 (Business Statistics), PSY 250 (Behavioral Statistics),
STAT 235 (Introduction to Statistics for Social Sciences), or STAT 241 (Elementary Statistics).   Additionally, take 
at least six credits from  
\begin{itemize}
\item \appliedstat
\item \statistics
\item \statisticsII
\end{itemize}
%\vspace{-0.1in}
%\subsubsection*{\textcolor{black}{Suggested mathematics course sequence}}
\statMinor



\end{document}
