\documentclass[10pt]{article}
\usepackage{utopia}
\usepackage{graphicx}
\usepackage{xcolor}
\usepackage{hyperref}
\usepackage{xspace}
\hypersetup{hidelinks}
\usepackage{comment}

\usepackage{enumitem}

\setlist{nosep}

\newenvironment{mypar}[2]
   {\begin{list}{}%
     {\setlength\leftmargin{#1}
     \setlength\rightmargin{#2}}
     \item[]}
   {\end{list}}

% see https://www.unk.edu/ccr/marketing-advertising/branding-and-identity-marks/files/UNK-graphics-standards-quick-guide.pdf

\definecolor{unkblue}{HTML}{002F6C}
\definecolor{unkgold}{HTML}{CC8A00}
\pagestyle{empty}

\usepackage[symbol]{footmisc}

\newcommand{\calcone}{\textbf{MATH 115} Calculus I with Analytic Geometry (fall and spring) \dotfill 5 credits}
\newcommand{\calconeshort}{MATH 115}

\newcommand{\calctwo}{\textbf{MATH 202} Calculus II with Analytic Geometry (fall and spring, prerequisite: MATH 115) \dotfill 5 credits }
\newcommand{\calctwoshort}{MATH 202}

\newcommand{\foundations}{\textbf{MATH 250} Foundations of Math (fall and spring, prerequisite: MATH 115)  \dotfill 3 credits}
\newcommand{\foundationsshort}{MATH 250}

\newcommand{\calcthree}{\textbf{MATH 260} Calculus III  (fall and spring, prerequisite: MATH 202) \dotfill 5 credits}
\newcommand{\calcthreeshort}{MATH 260}

\newcommand{\linear}{\textbf{MATH 280} Linear Algebra (spring only, prerequisite: MATH 115) \dotfill 3 credits}
\newcommand{\linearshort}{MATH 280}

\newcommand{\discrete}{\textbf{MATH 413} Discrete Mathematics  (fall only, prerequisite: MATH 250)\dotfill 3 credits}
\newcommand{\discreteshort}{MATH 413}

\newcommand{\statistics}{\textbf{STAT 441} Probability and Statistics (spring only, prerequisite: MATH 260)  \dotfill  3 credits}
\newcommand{\statisticsshort}{STAT 441}

\newcommand{\diffeq}{\textbf{MATH 305}	Differential Equations (spring only, prerequisite: MATH 260) \dotfill 	3 credits}
\newcommand{\diffeqshort}{MATH 305}

\newcommand{\abstractalgebra}{\textbf{MATH 350}	Abstract Algebra (spring only, prerequisite: MATH 250) \dotfill 	3 credits}
\newcommand{\abstractalgebrashort}{MATH 350}

\newcommand{\complex}{\textbf{MATH 365}	Complex Analysis (spring only,  prerequisite: MATH 260) \dotfill 3 credits}
\newcommand{\complexshort}{MATH 365}

\newcommand{\advancedcalc}{\textbf{MATH 460}	Advanced Calculus I  (fall only,   prerequisite: MATH 250) \dotfill 3 credits}
\newcommand{\advancedcalcshort}{MATH 460}

\newcommand{\numerical}{\textbf{MATH 420}   Numerical Analysis   (spring only, prerequisite: MATH 202)\dotfill 3 credits}
\newcommand{\numericalshort}{MATH 420} 

\newcommand{\collegegeometry}{\textbf{MATH 310}	College Geometry (fall only,  prerequisite: MATH 250) \dotfill 3 credits}
\newcommand{\collegegeometryshort}{MATH 310}

\newcommand{\mathhistory}{\textbf{MATH 400} History of Mathematics (fall only,  prerequisite: MATH 115) \dotfill 3 credits}
\newcommand{\mathhistoryshort}{MATH 400}

\newcommand{\numbertheory}{\textbf{MATH 404} Theory of Numbers (spring only,  prerequisite: MATH 250) \dotfill 3 credits}
\newcommand{\numbertheoryshort}{MATH 404} 

\newcommand{\appliedstat}{\textbf{STAT 345} Applied Statistics I (fall only, prerequisite: MATH 123 or MATH 115) \dotfill 3 credits}
\newcommand{\appliedstatshort}{STAT 345}

\newcommand{\physics}{\textbf{PHYS 275}  (corerequisite: \calconeshort) \dotfill 3 credits}

\newcommand{\contactbw}{\mbox{Barton Willis, PhD}, Department of Mathematics and Statistics,  Discovery Hall, Room 368;
email or phone: \href{mailto:willisb@unk.edu}{willisb@unk.edu} or 308-865-8868.}

\newcommand{\forinfo}[2]{If you would like to discuss the possibility of adding a math {#1} to your {#2}, please contact \contactbw}

\newcommand{\catalog}{2023--2024 }

\newcommand{\LOPER}{LOPER\xspace}
\newcommand{\myfootnote}{\footnote{This plan is up to date for  the \catalog catalog. The designation of a fall or spring class is 
anticipated, but  is subject to change (updated \today).}}

\newcommand{\uptodate}{This plan is up to date for  the \catalog catalog. The designation of a fall or spring class is 
anticipated, but  is subject to change.}
\newcommand{\myheading}{
\begin{flushleft}
\includegraphics[scale=0.35]{unk-logo}\\
\setcounter{footnote}{0}
\vspace{0.25in}
 \textcolor{unkblue}{Department of Mathematics and Statistics, Discovery Hall} \\
 %  \emph{\textcolor{unkblue}{Discovery Hall}}\\
  \textcolor{unkblue}{University of Nebraska at Kearney}
\end{flushleft}}

\newcommand{\bt}[1]{\textbf{#1}}


\newcommand{\mathminor}{
  \begin{center}
   \begin{tabular}[h]{| l | l | l|} 
      \hline
      & \textbf{Fall}         &  \textbf{Spring}  \\ \hline 
      \textbf{First Year} & \calconeshort & \calctwoshort, \linearshort \\  \hline
      \textbf{Second Year} &  \appliedstatshort{} &  \foundationsshort \\ \hline
      \textbf{Third Year} &    \discreteshort              &  \\ \hline
      \end{tabular}
      \end{center}}

\newcommand{\mathminorALT}{
\begin{center}
   \begin{tabular}[h]{| l | l | l|} 
      \hline
                 & \textbf{Fall}         &  \textbf{Spring}  \\ \hline 
      \textbf{First Year} & \calconeshort & \calctwoshort{}  \\  \hline
      \textbf{Second Year} &  \calcthreeshort{},  \foundationsshort{}  & \diffeqshort \\ \hline
         \end{tabular}
      \end{center}}
      
\newcommand{\mathminorALTALT}{
\begin{center}
   \begin{tabular}[h]{| l | l | l|} 
      \hline
                 & \textbf{Fall}         &  \textbf{Spring}  \\ \hline 
      \textbf{First Year} & \calconeshort & \calctwoshort{}  \\  \hline
      \textbf{Second Year} &  \foundationsshort{}  &  math elective \\ \hline
    \textbf{Third Year} &  math elective   &  math elective  (if needed) \\ \hline
         \end{tabular}
      \end{center}}
      
\newcommand{\mathBS}{
     \begin{center}
         \begin{tabular}[h]{| l | l | l|} 
            \hline
                       & \textbf{Fall}         &  \textbf{Spring}  \\ \hline 
            \textbf{First Year} & \calconeshort{}  & \calctwoshort \\  \hline
            \textbf{Second Year} &  \calcthreeshort{}, \foundationsshort & \diffeqshort, math elective \\ \hline
            \textbf{Third Year} & \advancedcalcshort              &  \complexshort{}, \linearshort \\ \hline
            \textbf{Fourth Year} & math elective &  \abstractalgebrashort  \\ \hline
         \end{tabular}
\end{center}}
\usepackage[final]{microtype}
\usepackage[american]{babel}
\usepackage[letterpaper, margin=0.65in]{geometry}


\begin{document}

\myheading


\subsection*{\textbf{\textcolor{unkblue}{Computer Science Comprehensive plus Math Minor}}}

Adding a math minor to your CS degree is one of the most beneficial 
combinations for a Computer Science major, especially if you are interested 
in machine learning, cryptography, or research. Your math minor will 
give you a good foundation in set theory, logic, and discrete probability, as 
well as providing you with a high level of mathematical understanding.


\forinfo{minor}{Computer Science Comprehensive degree}

\uptodate

\vspace{-0.1in}

\subsubsection*{\textcolor{black}{For your General Studies \LOPER 4 requirement take}}
\begin{itemize}
\item  \calcone
\end{itemize}

\subsubsection*{\textcolor{black}{To complete the Mathematics Minor take}}
\begin{itemize}
\item \calctwo
\item \foundations
\item \linear
\item \discrete
\item \appliedstat
\end{itemize}
\vspace{0.1in}
These are the classes that we recommend for a computer science major.
Your academic advisor can assist you with choosing alternatives 
to \linearshort, \discreteshort, and \appliedstatshort.

Once you declare a mathematics minor, your academic advisor will be happy to help you build a 
four-year plan for earning a minor in mathematics that fits with 
your other classes.  If you start on a path toward earning a math minor, but later decide to 
earn a  Mathematics Bachelor of 
Science, all the mathematics and statistics  classes listed here 
will count toward the Bachelor of Science degree.

\vspace{-0.1in}
\subsubsection*{\textcolor{black}{Suggested mathematics course sequence}}

\begin{center}
\mathminor
\end{center}


 


\newpage

\myheading



\vspace{-0.1in}
\subsection*{\textbf{\textcolor{unkblue}{Computer Science Comprehensive plus Mathematics Bachelor of Science}}}

Adding a math major to your CS degree is one of the most beneficial combinations for 
a Computer Science degree, especially if you are interested in machine learning, 
cryptography,   or research.  Your math major  will give you a solid foundation 
in set theory, logic, and discrete probability, 
as well providing you with a high level of mathematical understanding. 



\forinfo{major}{Computer Science Comprehensive degree}

\uptodate

\subsubsection*{\textcolor{black}{For your General Studies \LOPER 4 requirement take}}
\begin{itemize}
\item \calcone
\end{itemize}

\subsubsection*{\textcolor{black}{For your \LOPER 8 requirement take}}

\begin{itemize}
   \item \physics 
\end{itemize}

\subsubsection*{\textcolor{black}{For your Program Specific Requirement take}}

\begin{itemize}
   \item \textbf{ENG 102}, Special Topics in Academic Writing and Research \dotfill 3 credits
\end{itemize}

	

\subsubsection*{\textcolor{black}{For your Math Core requirements take}}

\begin{itemize}
   \item \calctwo
   \item \foundations
   \item \linear
   \item \calcthree
   \item \diffeq
  \item \abstractalgebra
  \item \complex
  \item \advancedcalc
\end{itemize}



\subsubsection*{\textcolor{black}{For your  Math Electives take}}
\begin{itemize}
\item \discrete
\item \statistics
\end{itemize}
\vspace{0.1in}
These are the classes that we recommend for a computer science major.
Your academic advisor can assist you with choosing alternatives 
to \discreteshort{} and \statisticsshort{}. Your academic advisor 
will be happy to help you build a four-year plan for earning a 
Mathematics Bachelor of Science degree. 



\subsubsection*{\textcolor{black}{Suggested mathematics course sequence}}
   \mathBS

\vfill 
\newpage

%----Cyber Security Operations Comprehensive, Bachelor of Science


\myheading

\subsection*{\textbf{\textcolor{unkblue}{Cyber Security Operations Comprehensive, Bachelor of Science plus Math Minor}}}

Adding a math minor to your CS degree is one of the most beneficial degree combinations 
for a Computer Science major, especially if you are interested in machine learning, 
cryptography,  or research.  Your math minor will give you a good foundation in 
set theory, logic, and discrete probability, as well providing you with a high level of 
mathematical understanding. This guide shows you how to choose the electives in your program to also 
earn a math minor. The nineteen additional credits needed to earn a mathematics minor 
will partially fulfill the 25 credits of unrestricted electives required by the 
Cyber Security Operations Comprehensive degree.

 
 \forinfo{major}{Cyber Security Operations Comprehensive}

\uptodate
\vspace{-0.1in}

\subsubsection*{\textcolor{black}{For  your General Studies \LOPER 4 requirement take}}
\begin{itemize}
\item  \calcone
\end{itemize}

\subsubsection*{\textcolor{black}{To complete the Mathematics Minor take}}

\begin{itemize}
\item \calctwo
\item \foundations
\item \linear
\item \appliedstat
\item \discrete
\end{itemize}
These are the classes that we recommend for a computer science major.
Your academic advisor can assist you with choosing alternatives 
to \appliedstatshort{} and \discreteshort{}.

Once you declare a mathematics minor, your academic advisor 
will be happy to help you build a four-year plan for earning a minor 
in mathematics that fits with your other classes.  If you start on 
a path toward earning a math minor, but latter decide to earn a  
Mathematics Bachelor of Science, all the mathematics and statistics  
classes listed here will count toward the Bachelor of Science degree.

\subsubsection*{\textcolor{black}{Suggested mathematics course sequence}}

\mathminor




\newpage

\myheading


\subsection*{\textbf{\textcolor{unkblue}{Physics Comprehensive plus Math Minor}}}

\noindent Success in Physics classes, especially  Analytic Mechanics (PHYS 402), Electricity \& Magnetism (PHYS 407), and Quantum Mechanics (PHYS 419) require a 
firm understanding of mathematics. That might help explain why 
nationwide about one-third of undergraduate physics majors also 
earn a degree in mathematics. If you choose to attend a graduate 
program in physics, earning a math minor will help you be successful 
with the math intense first year core physics graduate classes.

The nine credits of mathematics classes beyond the requirements for 
your Physics Comprehensive will count toward your required 
20 credits of unrestricted electives.


\forinfo{minor}{Physics Comprehensive}

\uptodate

\subsubsection*{\textcolor{black}{For your General Studies \LOPER 4 requirement take}}
\begin{itemize}
\item \calcone
\end{itemize}

\subsubsection*{\textcolor{black}{For your BS Science-related course requirements take}}
\begin{itemize}
\item \calctwo
\item \diffeq
\end{itemize}


\subsubsection*{\textcolor{black}{Physics Comprehensive Math Requirements take}}
\begin{itemize}
 \item \calcthree
\end{itemize}

\subsubsection*{\textcolor{black}{For your Math minor, take}}
\begin{itemize}
\item \foundations
\end{itemize}

\vspace{0.1in}
  

 Once you declare a mathematics minor, your academic advisor will be 
 happy to help you build a four-year plan. If you start on a path toward earning 
 a math minor, but latter decide to earn a  Mathematics Bachelor of Science, 
 all the math classes listed here will count toward the Bachelor of Science degree.


\subsubsection*{\textcolor{black}{Suggested mathematics course sequence}}

\mathminorALT




\newpage

\myheading

\subsection*{\textbf{\textcolor{unkblue}{Adding a Math Minor to your Major}}}

As you may know, at UNK every Bachelor of Science program of study requires a second
major or minor. So supplementing your major with a math minor is not only a great
way to build your r\'esum\'e, but it also fulfills a degree requirement.

By choosing to take Calculus I with Analytic Geometry (five credits) to satisfy
your \LOPER 4 General Studies requirement, you will gain five credits out of a
required twenty-four for a mathematics minor. That leaves only nineteen credits
to complete a math minor.  Of these remaining credits, six are electives, allowing
you to choose from classes ranging from History of Mathematics to Statistics.


\forinfo{minor}{program of study}

\uptodate

\subsubsection*{\textcolor{black}{For your General Studies \LOPER 4 requirement take}}
\begin{itemize}
\item \calcone
\end{itemize}


\subsubsection*{\textcolor{black}{For your Math minor, take}}
\begin{itemize}
  \item \calctwo

\item \foundations
\end{itemize}

\subsubsection*{\textcolor{black}{For your Math minor electives take}}

From the following, take \emph{at least eight} credits from
\vspace{0.1in}

\begin{itemize}
\item \linear
  \item \calcthree
\item \collegegeometry
\item \abstractalgebra
\item \numbertheory
\item \discrete
\item \advancedcalc
\item \diffeq
\item \appliedstat
\item \complex
\item \mathhistory
\item \numerical
\item \statistics
\end{itemize}

\subsubsection*{\textcolor{black}{Suggested mathematics course sequence}}

\mathminorALTALT

 %\noindent Once you declare a mathematics minor, your academic advisor will be happy to help you build a four-year plan for earning a minor in %mathematics.  If you start on a path toward earning a math minor, but latter decide to earn a  Mathematics Bachelor of Science, all the math classes %listed here will count toward the Bachelor of Science degree.

 %  \vspace{0.1in}

\newpage

%----------------
\myheading


\subsection*{\textbf{\textcolor{unkblue}{Adding a Math Major to your Major}}}

As you may know, at UNK every Bachelor of Science program of study requires a second
major or minor. So supplementing your  program of study with a math major is not only a great
way to build your r\'esum\'e, but it also fulfills a degree requirement.


\forinfo{major}{program of study}

\uptodate

\subsubsection*{\textcolor{black}{For your General Studies \LOPER 4 requirement take}}
\begin{itemize}
\item \calcone
\end{itemize}


\subsubsection*{\textcolor{black}{For your Math major requirements take}}
\begin{itemize}
  \item \calctwo
  \item \calcthree
  \item \linear
\item \foundations
\item \diffeq
\item \abstractalgebra
\item \complex

\item \advancedcalc
\end{itemize}

\subsubsection*{\textcolor{black}{For your Math major electives take}}

From the following, take \emph{at least six} credits from
\vspace{0.1in}

\begin{itemize}
\item \collegegeometry
\item \mathhistory
\item \numbertheory
\item \discrete
\item \numerical
\item \statistics
\end{itemize}

Once you declare a mathematics major, your academic advisor 
will be happy to help you build a four-year plan for earning a major 
in mathematics.

\subsubsection*{\textcolor{black}{Suggested mathematics course sequence}}

\mathBS



\end{document}
